\documentclass[15pt,a4paper]{article}

% 中文支持(只需要一次)
\usepackage[UTF8]{ctex}
\usepackage[a4paper,margin=2.5cm]{geometry}
\usepackage{graphicx}  % 插图
\usepackage{fancyhdr}
\usepackage{enumitem}
\usepackage{setspace}

% 页眉页脚设置
\pagestyle{fancy}

% ======== 修改以下信息即可 ========
\newcommand{\AuthorName}{刘皓中}
\newcommand{\StudentID}{3240111539}
\newcommand{\PaperTitle}{从“星链”计划看太空信息主权对国家空间安全的威胁与防御}
% ==================================

% 页眉样式:居中显示作者与标题,右侧页码
\fancyhf{}
\fancyhead[C]{\AuthorName \quad《\PaperTitle》}
\fancyhead[R]{\thepage}

% =============== 段落格式 ===================
\setlength{\parindent}{2em}   % 首行缩进2个汉字
\setlength{\parskip}{0.5em}   % 段间距



\begin{document}

% 标题部分 
\vspace*{0cm}
\begin{center}
\begin{spacing}{2.3}  % 局部1.5倍行距
    {\heiti\zihao{-2}\PaperTitle} \\[1em]
    {\kaishu\zihao{4}\AuthorName\quad 学号:\StudentID}
\end{spacing}
\end{center}


% 摘要部分
\noindent
{\heiti\zihao{-5}\textbf{摘 要:}}%
{\kaishu\zihao{-5}hhh本文主要探讨了人工智能技术在教育领域中的实际应用,包括个性化学习、自动化评估与智能辅导系统等方面的发展趋势与挑战。}\\
{\heiti\zihao{-5}\textbf{关键词:}}%
{\kaishu\zihao{-5}人工智能;教育应用;个性化学习;智能辅导}

% 英文部分
\vspace{2em}  % 与上文摘要之间留点空白

\begin{center}
    {\bfseries\zihao{4}Title} \\[1em]  % Title:Times New Roman,加粗,小四
    {\zihao{-4}\rmfamily Name: Haozhong Liu \quad Student ID: 3240111539}  % Name 和学号,小四,不加粗
\end{center}

\vspace{1.5em}

% 英文摘要
\noindent
{\bfseries\zihao{-5} Abstract: }%
{\rmfamily\zihao{-5} This paper explores the practical applications of artificial intelligence in education, including personalized learning, automated assessment, and intelligent tutoring systems. It also discusses the challenges and future trends in integrating AI technologies into modern education.}\\
{\bfseries\zihao{-5} Key Words: }%
{\rmfamily\zihao{-5} Artificial Intelligence; Educational Applications; Personalized Learning; Intelligent Tutoring}

\vspace{2em}

% ========== 正文部分 ==========

% 一级标题
\vspace{1em}
{\heiti\zihao{4}\noindent\textbf 1 \quad 引言}
\vspace{0.5em}

% 二级标题
{\heiti\zihao{-5}\noindent\textbf 1.1 \quad 研究背景}
\vspace{0.3em}

\songti\zihao{-5}
星链计划的出现标志着全球卫星互联网与太空竞争进入新阶段。随着航天商业化与信息化的加速,太空已成为继陆、海、空、网之后的第五安全领域。太空信息主权的争夺直接影响\textbf{国家空间安全(National Space Security)}与综合国力竞争格局。

\vspace{0.8em}

{\heiti\zihao{-5}\noindent\textbf 1.2 \quad 研究意义}
\vspace{0.3em}

\songti\zihao{-5}
从空间安全的视角探讨星链计划的战略风险,有助于完善国家安全体系的“太空防线”。此外,这对构建中国特色空间安全观、保障信息与主权安全具有重要的现实意义。

\vspace{0.8em}

{\heiti\zihao{-5}\noindent\textbf 1.3 \quad 研究思路与结构}
\vspace{0.3em}

\songti\zihao{-5}
研究方法主要包括案例分析法、战略分析法与比较分析法。论文结构如下所示: 第二章分析星链计划及其对空间安全格局的重塑; 第三章聚焦太空信息主权冲突下的国家空间安全威胁; 第四章提出空间安全防御体系构建的路径; 最后总结研究结论与未来展望。

\vspace{1em}

% 一级标题
{\heiti\zihao{4}\noindent\textbf 2 \quad 星链计划与空间安全格局的重塑}
\vspace{0.5em}

% 二级标题
{\heiti\zihao{-5}\noindent\textbf 2.1 \quad 星链计划的技术与体系特点}
\vspace{0.3em}

% 三级标题
{\kaishu\zihao{-5}\noindent 2.1.1 \quad 卫星数量、轨道分布与网络架构}
\vspace{0.2em}

\songti\zihao{-5}
星链计划计划部署超过四万颗低轨卫星,形成全球范围的高速网络覆盖。其多层轨道分布与网状互联架构使得通信延迟显著降低,具备强大的全球接入能力与灵活的网络路由机制。

\vspace{0.5em}

{\kaishu\zihao{-5}\noindent 2.1.2 \quad 系统特征:全球覆盖、低延迟、高抗毁性}
\vspace{0.2em}

\songti\zihao{-5}
星链系统通过卫星间激光链路实现全球高速互联,具备低延迟和强抗毁特征,可在地面基础设施遭受攻击或损毁时维持通信功能。这使其具备显著的军事与战略价值。

\vspace{0.5em}

{\kaishu\zihao{-5}\noindent 2.1.3 \quad “星链”在空间信息传输中的战略地位}
\vspace{0.2em}

\songti\zihao{-5}
星链不仅是一种商业卫星互联网系统,更是空间信息传输基础设施的战略节点。其全球数据中继能力可被用于情报收集、战场指挥与空间监控,对国家空间安全格局产生深远影响。

\vspace{1em}

% 二级标题
{\heiti\zihao{-5}\noindent\textbf 2.2 \quad 星链与美国空间安全战略}
\vspace{0.3em}

\songti\zihao{-5}
从“国家太空政策”(National Space Policy)到“星链”体系,美国在空间安全领域形成了系统的延伸逻辑。其核心是军民融合(civil-military integration)与太空控制(space control)的双重目标。星链已成为美国实施全球空间安全战略与维持信息霸权的重要支撑平台。

\vspace{0.8em}

% 二级标题
{\heiti\zihao{-5}\noindent\textbf 2.3 \quad 对全球空间安全格局的影响}
\vspace{0.3em}

\songti\zihao{-5}
星链的部署打破了传统空间力量的均衡,加速了“太空军事化”与“太空信息主权化”的趋势,对其他国家的空间安全战略构成外部压力。其广域覆盖与抗干扰能力可能在未来战争中成为决定性因素。

\vspace{1em}

% 插入图片
\begin{figure}[htbp]
    \centering
    \includegraphics[width=0.65\textwidth]{example.png}
    \caption{星链计划与空间安全格局示意图}
    \label{fig:starlink_security}
\end{figure}

\vspace{1em}

% 一级标题
{\heiti\zihao{4}\noindent\textbf 3 \quad 太空信息主权冲突下的国家空间安全威胁}
\vspace{0.5em}

% 二级标题
{\heiti\zihao{-5}\noindent\textbf 3.1 \quad 太空信息主权的战略内涵}
\vspace{0.3em}

\songti\zihao{-5}
外层空间的信息主权不仅关乎通信与数据安全,更与国家空间安全密切相关。掌握太空信息主权意味着在轨数据流、空间态势感知与卫星网络控制上的主导权。

\vspace{0.8em}

{\heiti\zihao{-5}\noindent\textbf 3.2 \quad 星链对国家空间安全的多维威胁}
\vspace{0.3em}

\songti\zihao{-5}
星链计划可能带来通信依赖、数据外泄、空间干扰与军事介入等多重风险。其私营性质与政府合作模式模糊了商业与军事界限,增加了外部势力干预空间领域的可能性。

\vspace{0.8em}

{\heiti\zihao{-5}\noindent\textbf 3.3 \quad 案例分析:俄乌冲突中的星链使用}
\vspace{0.3em}

{\kaishu\zihao{-5}\noindent 3.3.1 \quad 星链在战时通信保障中的作用}
\vspace{0.2em}

\songti\zihao{-5}
在俄乌冲突中,星链成为乌方关键的通信支撑系统,确保了前线部队的指挥与情报传输能力,体现出其军事通信潜力。

\vspace{0.5em}

{\kaishu\zihao{-5}\noindent 3.3.2 \quad 对空间安全与信息主权界限的冲击}
\vspace{0.2em}

\songti\zihao{-5}
星链的军事化使用模糊了民用与军用的界限,对国际法和太空主权原则造成冲击,揭示出太空信息主权缺乏有效约束的现实困境。

\vspace{0.5em}

{\kaishu\zihao{-5}\noindent 3.3.3 \quad 中国空间安全的现实启示}
\vspace{0.2em}

\songti\zihao{-5}
该案例提示中国需警惕外部太空信息网络的潜在控制风险,加快构建自主可控的卫星互联网体系,提升空间通信安全防御能力。

\vspace{1em}

% 一级标题
{\heiti\zihao{4}\noindent\textbf 4 \quad 国家空间安全防御体系构建}
\vspace{0.5em}

{\heiti\zihao{-5}\noindent\textbf 4.1 \quad 战略层面:完善国家空间安全体系}
\vspace{0.3em}

\songti\zihao{-5}
应将空间安全纳入国家总体安全观,完善顶层设计与政策体系,建立国家空间安全协调机制,推动太空资源与信息主权一体化管理。

\vspace{0.8em}

{\heiti\zihao{-5}\noindent\textbf 4.2 \quad 技术层面:增强空间安全防御能力}
\vspace{0.3em}

\songti\zihao{-5}
加强低轨通信卫星、自主导航与空间监测技术研发,构建空间信息防护体系,包括抗干扰通信、数据加密与太空态势感知系统。

\vspace{0.8em}

{\heiti\zihao{-5}\noindent\textbf 4.3 \quad 国际层面:参与全球空间安全治理}
\vspace{0.3em}

\songti\zihao{-5}
积极推动国际太空规则制定与多边治理,倡导“和平利用、共治共享”的空间安全理念,构建公平开放的太空信息秩序。

\vspace{1em}

% 一级标题
{\heiti\zihao{4}\noindent\textbf 5 \quad 结论}
\vspace{0.5em}

{\heiti\zihao{-5}\noindent\textbf 5.1 \quad 研究结论}
\vspace{0.3em}

\songti\zihao{-5}
星链计划的扩张在实质上构成了对国家空间安全的系统性威胁。太空信息主权已成为影响空间安全的新焦点。国家应从战略、技术与国际三个维度共同构建空间安全防御体系。

\vspace{0.8em}

{\heiti\zihao{-5}\noindent\textbf 5.2 \quad 未来展望}
\vspace{0.3em}

\songti\zihao{-5}
未来太空安全竞争将进入“信息主权—网络体系—战略防御”三位一体的新阶段。中国应在空间安全领域确立话语权与制度主导权,以维护国家空间利益与全球战略稳定。

\vspace{1.5em}


% ========== 参考文献 ==========
\noindent
{\songti\zihao{-5}\textbf{参考文献}}
\vspace{0.5em}

\begin{enumerate}[label={[\arabic*]}, leftmargin=2.8em, labelwidth=2.2em, labelsep=0.6em, itemsep=0.6em, align=left]
  \item {\kaishu\zihao{-5} 张三, 李四. 人工智能在教育领域的应用研究[J]. 中国教育技术, 2024, 38(2): 45-52.}
  \item {\kaishu\zihao{-5} 王明华. 智能教育系统设计与实现[M]. 北京: 清华大学出版社, 2023. 78-95.}
  \item {\kaishu\zihao{-5} Chen L, Smith J. AI-driven personalized learning: A comprehensive review[J]. Educational Technology Research and Development, 2024, 72(1): 123-145.}
  \item {\kaishu\zihao{-5} 赵丽娟. 基于深度学习的智能辅导系统研究[D]. 博士学位论文, 北京师范大学, 2023.}
  \item {\kaishu\zihao{-5} 教育部. 人工智能助推教师队伍建设行动计划[EB/OL]. http://www.moe.gov.cn, 2024-03-15.}
\end{enumerate}







\end{document}
